% mnras_template.tex 
%
% LaTeX template for creating an MNRAS paper
%
% v3.0 released 14 May 2015
% (version numbers match those of mnras.cls)
%
% Copyright (C) Royal Astronomical Society 2015
% Authors:
% Keith T. Smith (Royal Astronomical Society)

% Change log
%
% v3.0 May 2015
%    Renamed to match the new package name
%    Version number matches mnras.cls
%    A few minor tweaks to wording
% v1.0 September 2013
%    Beta testing only - never publicly released
%    First version: a simple (ish) template for creating an MNRAS paper

%%%%%%%%%%%%%%%%%%%%%%%%%%%%%%%%%%%%%%%%%%%%%%%%%%
% Basic setup. Most papers should leave these options alone.
\documentclass[fleqn,usenatbib]{mnras}

% MNRAS is set in Times font. If you don't have this installed (most LaTeX
% installations will be fine) or prefer the old Computer Modern fonts, comment
% out the following line
\usepackage{newtxtext,newtxmath}
% Depending on your LaTeX fonts installation, you might get better results with one of these:
%\usepackage{mathptmx}
%\usepackage{txfonts}

% Use vector fonts, so it zooms properly in on-screen viewing software
% Don't change these lines unless you know what you are doing
\usepackage[T1]{fontenc}
\usepackage{subfig} %for figure subplot
\usepackage{graphicx}
\usepackage[flushleft]{threeparttable}
\usepackage{xspace}



%%%%%%%%%%%%%%%%%%%%%%%%%%%%%%%%%%%%%%%%%%%%%%%%%%%%%%%%%%%%%%%%%%%
% my commands
\newcommand{\lcdm}{$\Lambda$CDM}
\newcommand{\hst}{{\it HST}}
\newcommand{\efr}{$R_{\mathrm{eff}}$}
\newcommand{\galfit}{{\sc Galfit}}
\newcommand{\mbh}{$\mathcal M_{\rm BH}$}
\newcommand{\lhost}{$L_{\rm host}$}
\newcommand{\mr}{$Mag_{\rm ~R}$}
\newcommand{\halpha}{H$_{\alpha}$}
\newcommand{\Hb}{H$_{\beta}$}
\newcommand{\sersic}{S\'ersic}
\newcommand{\lenstronomy}{{\sc Lenstronomy}}
\newcommand{\reff}{{$R_{\mathrm{eff}}$}}
\newcommand{\kmsMpc}{km~s$^{\rm -1}$~Mpc$^{\rm -1}$}
\newcommand{\kms}{\ifmmode{\,\rm{km}\, \rm{s}^{-1}}\else{$\,$km$\,$s$^{-1}$}\fi}
\newcommand{\sigstar}{{$\sigma_*$}}
\newcommand{\mstar}{{$M_*$}}
\newcommand{\Mgii}{Mg$_{\rm II}$}
\newcommand{\Civ}{C$_{\rm IV}$}
%\newcommand{\farcs}{\mbox{\ensuremath{.\!\!^{\prime\prime}}}}% fractional arcsecond symbol: 0.''0
\newcommand{\sam}{\texttt{SAM}}
\newcommand{\mbii}{\texttt{MBII}}
%%%%%%%%%%%%%%%%%%%%%%%%%%%%%%%%%%%%%%%%%%%%%%%%%%%%%%%%%%%%%%%%%%%
\newcommand{\ding}[1]{\textcolor{red}{[{\bf Xuheng}: #1]}}
\newcommand{\todo}[1]{\textcolor{red}{[{\bf Todo}: #1]}}  
\newcommand{\red}[1]{{ \textcolor{red}{#1}}}
\newcommand{\blue}[1]{{ \textcolor{blue}{#1}}}
\newcommand{\pink}[1]{{ \textcolor{magenta}{#1}}}

% Allow "Thomas van Noord" and "Simon de Laguarde" and alike to be sorted by "N" and "L" etc. in the bibliography.
% Write the name in the bibliography as "\VAN{Noord}{Van}{van} Noord, Thomas"
\DeclareRobustCommand{\VAN}[3]{#2}
\let\VANthebibliography\thebibliography
\def\thebibliography{\DeclareRobustCommand{\VAN}[3]{##3}\VANthebibliography}


%%%%% AUTHORS - PLACE YOUR OWN PACKAGES HERE %%%%%

% Only include extra packages if you really need them. Common packages are:
\usepackage{graphicx}	% Including figure files
\usepackage{amsmath}	% Advanced maths commands
\usepackage{amssymb}	% Extra maths symbols

%%%%%%%%%%%%%%%%%%%%%%%%%%%%%%%%%%%%%%%%%%%%%%%%%%

%%%%% AUTHORS - PLACE YOUR OWN COMMANDS HERE %%%%%

% Please keep new commands to a minimum, and use \newcommand not \def to avoid
% overwriting existing commands. Example:
%\newcommand{\pcm}{\,cm$^{-2}$}	% per cm-squared

%%%%%%%%%%%%%%%%%%%%%%%%%%%%%%%%%%%%%%%%%%%%%%%%%%

%%%%%%%%%%%%%%%%%%% TITLE PAGE %%%%%%%%%%%%%%%%%%%

% Title of the paper, and the short title which is used in the headers.
% Keep the title short and informative.
\title[Mass relations by lensed AGN hosts]{The Mass Relations between Supermassive Black Holes and Their Host Galaxies by Eight Strongly Lensed AGN Systems.}

% The list of authors, and the short list which is used in the headers.
% If you need two or more lines of authors, add an extra line using \newauthor
\author[X. Ding et al.]{
Xuheng Ding,$^{1}$\thanks{E-mail: dxh@astro.ucla.edu}
Tommaso Treu,$^{1}$
Simon Birrer,$^{1, 2}$
Dominique Sluse,\newauthor
Chris Fassnacht,
Matthew W. Auger,
Sherry H. Suyu,
Kenneth C. Wong,\newauthor
Takahiro Morshita,
and et al. $^{3}$
\\
% List of institutions
$^{1}$Department of Physics and Astronomy, University of California, Los Angeles, CA, 90095-1547, USA\\
$^{2}$Kavli Institute for Particle Astrophysics and Cosmology and Department of Physics, Stanford University, Stanford, CA 94305, USA\\
$^{3}$xxx
}

% These dates will be filled out by the publisher
\date{Accepted XXX. Received YYY; in original form ZZZ}

% Enter the current year, for the copyright statements etc.
\pubyear{2020}

% Don't change these lines
\begin{document}
\label{firstpage}
\pagerange{\pageref{firstpage}--\pageref{lastpage}}
\maketitle

% Abstract of the paper
\begin{abstract}
Strong lensed AGN has been considered as a unique tool to exam the correlations between the mass of the supermassive Black Hole and its host galaxies. In this work, we have adopted eight strongly lensed system from the H0LiCOW collaboration. The deep \hst\ data and adopt the \lenstronomy\ to reconstruct the image of the source. The \mbh\ are estimated using the board emission line. We compare our inference to the once in the literature and find a consistent result. Combining them together, we are able to constrain the $\gamma$ to xxx. Our our demonstrate is demonstrate the power of using strong lensed AGNs. The sample of the data are supposed to increase rapidly.
\end{abstract}

% Select between one and six entries from the list of approved keywords.
% Don't make up new ones.
\begin{keywords}
galaxies: evolution -- galaxies: active -- gravitational lensing: strong
\end{keywords}

%%%%%%%%%%%%%%%%%%%%%%%%%%%%%%%%%%%%%%%%%%%%%%%%%%

%%%%%%%%%%%%%%%%% BODY OF PAPER %%%%%%%%%%%%%%%%%%

\section{Introduction}
Structure:

[] What is scaling relations.

[] Strong lensing provide a tool to study to higher redshift.

[] Previous work. H0LiCOW XI. H0LiCOW XII. In these two works, the host galaxy is first reconstructed using the pixellation. Then, the host image are inferred in the source plan by fitting as a \sersic\ model. This means, the two-step inference of the host information. 

[] In this work, we adopt an independent tool and carry out a direct inference of the host properties. 

[]This paper is organized as follows. Zeropoint are defined in the AB systems.

\section{Sample Selection}
We adopt the eight lens systems from our H0LiCOW collaboration including HE~0435$-$1223, RXJ~1131$-$1231, WFI~2033$-$4723, HE~1104$-$1805, SDSS~1206$+$4332, SDSS~0246$-$0825, HE~0047$-$1756 and HS~2209$+$1914. For conciseness, we abbreviate each lens name to four digits (e.g., RXJ~1131$-$1231 to RXJ1131). In \citet{Ding2017a}, the simulation exercise has been performed to understand the fidelity of the source reconstruction based on these eight systems and verified that the host inference is trustworthy when its magnitude are brighter than 20 magnitude, 2$-$4 magnitudes dimmer than the AGN. The detailed information for the information of these eight systems are given in Table~1 therein. Besides, half of the systems, including HE0435, RXJ1131, WFI2033 and HE1104 have \hst\ imaging data in multi-bands. To estimate the \mbh, we adopt the spectra information as inferred by \citet{Sluse2012, Peng2006, Shen2011}. We use a Table[] to summarize the data information. 

While we note that our sample size included eight systems, which is a limited sample to constrain the evolution of the scaling relation. In this particular work, we simply aim to plot the scaling relation of our sample and make comparison to the ones in the literature. %[]The comparison sample (32 QSO paper)
We adopt the recent efforts by~\citet{Ding2020}, including 32 AGNs at redshift range $1.2<z<1.9$, 79 redshift QSO and 55 local measurements.

\begin{table}
\centering
%\begin{threeparttable}
\caption{Summary of lensed AGN information.}\label{data_set}
\resizebox{9cm}{!}{
     \begin{tabular}{cccccc}
        \hline
Object ID & $z$ & camera & Filter & exposure & pixel scale \\
 & & & & time (s) & (drizzled) \\ \hline
HE0435$-$1223 & 1.69 & WFC3-IR & F160W & 9340 & $0\farcs{}08$ \\
RXJ1131$-$1231 & 0.65 & ACS & F814W & 1980 & $0\farcs{}05$ \\
WFI2033$-$4723 & 1.66 & WFC3-IR & F160W & 26257 & $0\farcs{}08$ \\
SDSS1206$+$4332 & 1.79 & WFC3-IR & F160W & 8457 & $0\farcs{}08$ \\
HE1104$-$1805 & 2.32 & WFC3-IR & F160W & 14698 & $0\farcs{}08$ \\
SDSS0246$-$0825 & 1.68 & WFC3-UVIS & F814W & 8481 & $0\farcs{}03$ \\
HS2209$+$1914 & 1.07 & WFC3-UVIS & F814W & 14238 & $0\farcs{}03$ \\
HE0047$-$1756 & 1.66 & WFC3-UVIS & F814W & 9712 & $0\farcs{}03$ \\
        \hline
     \end{tabular}
    }
\begin{tablenotes}
      \small
      \item Note: $-$ The observational information is presented in~\citet{Ding2017a}.
\end{tablenotes}  
%\end{threeparttable}
\end{table}

To assure the validity of the comparison, we adopt a class of self-consistent recipes as introduced in~\citet{Ding2020} to estimate the \mbh\ of our sample. For the overall systems, we adopt the following virial formalism:
\begin{eqnarray}
\label{recipe}
\log \left(\frac{\mathcal M_{\rm BH}}{M_{\odot}}\right)&~=~& a+b \log \left(\frac{ \rm L _{\lambda_{line}}}{10^{44}{\rm erg~s^{-1}}}\right) \nonumber\\
&~+~& 2 \log \left(\frac{\rm FWHM(line)}{1000 ~{\rm km~s^{-1}}}\right) , 
\end {eqnarray}
%
with a\{\Civ, \Mgii, \Hb\}=\{6.322, 6.623, 6.910\},
b\{\Civ, \Mgii, \Hb\}=\{0.53, 0.47, 0.50\},
$\lambda_{line}$\{\Civ, \Mgii, \Hb\}=\{1350, 3000, 5100\}.
%
Having achieved a consistent cross-calibration, we adopt the board line information from the literature to estimate the \mbh. The information are presented in Table~\ref{mbh}. The uncertainty of the \mbh\ are expected as $0.4$~dex.


\begin{table}
\centering
%\begin{threeparttable}
\caption{Summary of lensed \mbh\ estimates.}\label{mbh}
\resizebox{8.5cm}{!}{
     \begin{tabular}{ccccc}
        \hline
Object ID & Line(s) used & FWHM & log($L_\lambda$) & $\log$\mbh \\
& & (\kms) & (${\rm erg~s^{-1}}$) & (M$_{\odot}$) \\ \hline
HE0435 & \Mgii & 4930 & 45.14 & 8.54 \\
RXJ1131 & \Mgii/\Hb & 5630/4545 & 44.29/44.02 & 8.26/8.23 \\
WFI2033 & \Mgii & 3960 & 45.19 & 8.38 \\
SDSS1206 & \Mgii & 5632 & 45.61 & 8.88 \\
HE1104 & \Civ & 6004 & 46.18 & 9.03 \\
SDSS0246 & \Mgii & 3700 & 45.19 & 8.32 \\
HS2209 & xxx & xxx & xxx &  xxx  \\
HE0047 & \Mgii & 4145 & 45.59 & 8.60 \\
        \hline
     \end{tabular}
    }
\begin{tablenotes}
      \small
      \item Note: $-$ The reference for the board line information. For SDSS1206, we follow Birrer2018~ref[] and consider the magnification effect to correct the L.
\end{tablenotes}  
%\end{threeparttable}
\end{table}

\section{surface photometry with lensing technique}
We present the lensed host galaxy source reconstruction to derive their intrinsic photometry in this section. Our lensed AGN systems are from the H0LiCOW project, and the lens models of four systems have been derived with host galaxies reconstructed ref[]. Aiming at time-delay cosmography, these works focused on deriving the precise and accurate lens models, while the reconstruction of the host galaxy is only a by-product. In this paper, we apply the state-of-the-art lens modeling technique and dedicate to recovering the photometry of the host galaxy. As will show, approach would allow us to obtain a one-step host magnitude defined by \sersic\ profile, which is consistent to the definitions in the literature. %Furthermore, for cross check purpose, we compare our inference to the reconstructions by the previous works.

\subsection{Model Choice}
We carry out the fitting process to subtract the central AGN light, constrain the lens model, and reconstruct the host galaxy in the source plan, simultaneously.
For all the systems, we assume the photometry of the galaxies (including the lensing galaxy and source galaxy) is described by the 2-D elliptical \sersic\ profile. We start with a single \sersic\ and consider to use two \sersic\ if any significant residual indicate a multiply component. For single \sersic\ case, we set the prior of the \sersic\ index value $n$ between $[0.5-5.0]$ to avoid unphysical inference\footnote{It has been studied in ref[] that this prior value would result in a consistent host magnitude inference as to fix n to its true value.}. The bright AGNs are unsolved and modelled by a scaled point spread function (PSF). We also joint the fitting of the AGN position to the center of its host.
 We adopt the elliptical Power-law model to describe the surface mass density of the deflector; an external shear component is also added.

We employ the imaging modelling tool \lenstronomy~ref[] to perform the fitting task. We following the standard procedure as described by~\citet{Ding2020} to prepare the fitting ingredients including the lensing imaging data, noise level map and the PSF information. The imaging data are drizzled to a higher resolution by setting a Gaussian kernel; the adopted final resolution are listed in Table~\ref{data_set}. We then adopt the {\sc photutils} by Python to model the 2-D global background light using SExtractor algorithm. We then remove the background light and cut the clear image data into postage stamp at a suitable size. We draw the image mask for each systems to define the region in which the pixels would to be included in the likelihood; see Figure~\ref{fig:image_inference}.

For each lens system, we adopt different choices to perform the fitting. The distribution of the fitting inference are assumed to sufficiently cover the truth. We then apply a statistical measurement to weighting for the final inference and estimate the uncertainty level. The different choice of the fitting including the following.
\begin{enumerate}
\item Following common practice, we select all the bright, isolated stars across the image frame of targets to define the PSF. Each selected star is consider as the initial PSF to input to the fitting, which is considered as different choice.
\item The central pixel of the AGNs are very bright and have systematic errors during the interpolation of the subsampled PSF. To avoid overfitting the noise, we adopt two modeling choices. 1)~manually boost the noise estimate in the central area to effectively infinite; 2)~perform the iterative PSF estimation as introduced by ref[].
\item To calculate the ray tracing under a higher resolution grid, the different relative level to the pixel sizes is selected, including [$2\times2$, $3\times3$] pixel$^{-1}$.
\item Using the lensing imaging alone is difficult to constrain the power-law slope. To avoid any overfitting, we fix the slope values during the lens modelling with three optional value $[1.9, 2.0, 2.1]$.
\end{enumerate}

All in all, for one lens system with a number of $N$ initial PSFs, we select in total of $N\times12$ (i.e., $2 \times2 \times3$) different choices of fitting. After  all sets of fittings are completed, we rank their performance of each choice based on their best-fitted $\chi^2$ value. Finally, we adopt the weighting algorithm as introduced by Equations (3)$-$(6) in~\citet{Ding2020} and combine the top eight choices to derive the final value and evaluate the uncertainties.

\subsection{Photometry Inference}
In this subsection, we describe the details of the fitting for each systems and present our inference of the host galaxy photometry.

\subsubsection{HE0435}
The modeling is set up based on the descriptions in the last section. We select five isolated stars in this field initial PSFs to input to the fitting with in total of 60 fittings for this system. Based on the top-eight choices, we perform the weighting algorithm and measure the host-to-total flux ratio, host magnitude, effective radius and \sersic\ index. The inference results are shown in Table~\ref{tab:host_measure}, (2)$-$(5) columns.

The lens model inference based on the best-fit result for HE0435 is drawn in Figure~\ref{fig:image_inference}-(a). Not surprisingly, we note that the residual level reveals are appears to be larger than the ones as presented by Ken[] or Simon[]. This is due to the fact that the definition of the host galaxy in our models is a 2-D \sersic\ model which is relatively simple compare to the pixellated reconstruction technique by {\sc Glee} or the shapelet functions as adopted by \lenstronomy. The smooth feature of our model couldn't capture the clumps in the host of the star-forming regions, however, our model is more effective to derive the global host light as function of the \sersic\ which is consistent to measurements in the comparison sample. 

As a cross check, we compare our host magnitude to the measurement in Ding+2017[], in which they model the reconstructed host galaxy as \sersic\ based on Galfit []. The host magnitude measured in that paper is: $mag = 21.75 \pm 0.13$%, $R_{eff} = 0.82 \pm 0.14$ and $n = 3.94 \pm 0.14$. 
which is slightly fainter than our results, i.e. $mag = 21.33 \pm 0.2$. \ding{Some discussion of the origin of this offset?}
Still, the center mismatch is only $0.15$~dex, which is a little of the total error budget.

Besides the F160W band, HE0435 has also been observed by the \hst\ through F814W and F555W. We carry out the similar steps to modeling at the two band. The inferred host magnitudes are xxx and xxx, respectively. Nevertheless, given that the data quality and the host flux is less bright in these two bands, we only combining the modeling lensed arcs in the image plane to inference the color of the host galaxy and we use {\sc Gsf} code to fit the color and find a 1.5 Gyr stellar template with solar metallicity could well match this color. This stellar template would be adopted to infer the \mstar\ in the next section.

\begin{figure*}
\centering
\begin{tabular}{c c}
\subfloat[HE0435]{\includegraphics[trim = 40mm 25mm 40mm 25mm, clip, width=0.5\textwidth]{fig/HE0435_best_inference.pdf}}&
\subfloat[RXJ1131]{\includegraphics[trim = 40mm 25mm 40mm 25mm, clip, width=0.5\textwidth]{fig/RXJ1131_best_inference.pdf}}\\
\subfloat[WFI2033]{\includegraphics[trim = 40mm 25mm 40mm 25mm, clip, width=0.5\textwidth]{fig/WFI2033_best_inference.pdf}}&
\subfloat[SDSS1206]{\includegraphics[trim = 40mm 25mm 40mm 25mm, clip, width=0.5\textwidth]{fig/SDSS1206_best_inference.pdf}}\\
\subfloat[HE1104]{\includegraphics[trim = 40mm 25mm 40mm 25mm, clip, width=0.5\textwidth]{fig/HE1104_best_inference.pdf}}&
\subfloat[SDSS0246]{\includegraphics[trim = 40mm 25mm 40mm 25mm, clip, width=0.5\textwidth]{fig/SDSS0246_best_inference.pdf}}\\
\subfloat[HS2209]{\includegraphics[trim = 40mm 25mm 40mm 25mm, clip, width=0.5\textwidth]{fig/HS2209_best_inference.pdf}}&
\subfloat[HE0047]{\includegraphics[trim = 40mm 25mm 40mm 25mm, clip, width=0.5\textwidth]{fig/HE0047_best_inference.pdf}}\\
\end{tabular}
\caption{\label{fig:image_inference} 
Illustrations of the inference for each system. On the top: (left to right): . On the bottom: (left to right.) }
\end{figure*} 

\subsubsection{RXJ1131}
The RXJ1131 has been well studied in many reference. The host galaxy of this system is lensed to a spacious arc with a boundary line between the dominated area of bulge and disk could be clearly seen. Thus, we set the host galaxy is composed of two \sersic\ with index value $n$ fixed to $1$ and $4$ to mimic the light distribution of the disk and bulge, respectively. Besides, same as suyu[], we consider the perturbations by the small object ($0\farcs{5}$ in the north) and fitting as the SIS and \sersic\ for its mass and light, respectively.

The inference of the RXJ1131 based on three initial PSFs are summarized in Table~\ref{tab:host_measure}, with the best-fit result drawn in Figure~\ref{fig:image_inference}-~(b).

We also compare our measurement to the previous reconstructions in Ding. Based on the reconstructions by Suyu[], the inference of the host magnitude by Ding[] is $mag_{\rm disk} = 20.07\pm0.06$ and $mag_{\rm bulge} = 21.81 \pm 0.28$. Compared to our inference in Table~\ref{tab:host_measure}, our inferred bulge flux is consistent at $\sim0.25$ dex level, however, while the inferred disk light is much brighter by $\sim 0.7$ dex.

Due to the fact that the arc is very extended in the image plane which is resolved to the AGNs, one could directly perform the SED fitting of the arc in the image plane. Combining the HST imaging data through F814W and F555W, Ding[] has inference the color and we adopt the stellar templates 3 Gyr and 1.5 Gyr with solar metallicity stellar to its disk and bulge, respectively.

\subsubsection{WFI2033}
WFI2033 is the last quads which has also been carefully studied recently in Ken[]. There is a satellite galaxy in the north. However, as noted by Ken[], it has a much smaller mass than the main deflector, thus we ignore its influence on the magnification but only fit its light using a \sersic\ model. There are in total of eight PSF stars are selected to modelling this system.

The final inference are presented in Table~\ref{tab:host_measure} and Figure~\ref{fig:image_inference}-~(c). We also compare the inference to the reconstructions by Ken. Fitting their reconstruction as a \sersic\ profile, we have inferred that $mag = 21.98 \pm 0.15$ which is slightly fainter than our inference by $\sim0.2$~dex.

Besides the F160W filter, there are also four other filters by F125W, F140W, F555W and F814W was available in the HST. We perform the lens model inference for all the filter and infer the host color through five band in the image plane. We find a stellar template with 0.625 Gyr could well match their color.

\subsubsection{SDSS1206}
SDSS1206 is a special system -- the AGN is a doubly images while most of the host is falls inside the inner caustic thus being quadruply imaged. We consider the galaxy triplet group at the north-west and use a single SIS model to denote their over all mass perturbations. Moreover, as noted by Simon[] (Figure 1 therein), a sub-clump is located in the north which is hardly visible. We model it as a SIS mass model and circular \sersic\ light model with joint centroids. Though we see the visible residuals in the fitted arcs using a single \sersic\ model, we find a double \sersic\ could help significantly improve the goodness of the fitting. Thus, we adopt the single \sersic\ model in our final inference. 

The results are presented in Table~\ref{tab:host_measure} and Figure~\ref{fig:image_inference}-~(d).
\todo{Ask Simon for the reconstruction image to compare.}

Beside F160W, there is no other \hst\ imaging data for this system and the color. For such case, we adopt the criteria by Ding2020[] to define the stellar template, i.e., the 1 Gyr and 0.625 Gyr stellar population would be adopted for systems as $z<1.44$ and $z>1.44$, respectively. Since the source redshfit of SDSS1206 is at 1.79, a 0.625 Gyr is selected.

\subsubsection{HE1104}

\subsubsection{SDSS0246}

\subsubsection{HE0047}

\subsubsection{HS2209}

The comparison to the previous fitting.

We summarized the inference for the eight systems in the table.
name--Setts--BestFitChisq--Totalflux-- Host ratio --Reff--Sersicn--magnitude - Stellar mass

\begin{table*}
\renewcommand{\arraystretch}{1.5}
%\setlength{\tabcolsep}{20pt}
\centering
  \begin{threeparttable}
\caption{Summary of lensed AGN inference.}\label{tab:host_measure}    
%\resizebox{12cm}{!}{
     \begin{tabular}{ccccccc}
\hline
Object ID & Magnitude & Host-Total Flux Ratio & Reff & \sersic\ $n$ & adopted AGE & $\log (M_{*}$)  \\
 & (AB system) & ($\%$) & (arcsec) & & (Gyr) & (M$_{\odot}$) \\ \hline
HE0435 & $21.332\substack{+0.208\\-0.175}$ & $62.471\pm10.893$ & $0.298\pm0.027$ & $2.829\pm0.251$ & $1.500$ & $10.98\substack{+0.07\\-0.08}$ \\
RXJ1131$_{\rm bulge}$ & $22.074\substack{+0.164\\-0.142}$ & $6.634\pm0.929$ & $0.106\pm0.007$ & fix to 4 & $3.000$ & $10.28\substack{+0.06\\-0.07}$ \\
RXJ1131$_{\rm disk}$ & $19.305\substack{+0.154\\-0.135}$ & $84.939\pm11.241$ & $0.875\pm0.062$ & fix to 1 & $1.500$ & $11.09\substack{+0.05\\-0.06}$ \\
WFI2033 & $21.404\substack{+0.167\\-0.145}$ & $35.723\pm5.093$ & $0.294\pm0.025$ & $0.528\pm0.011$ & $0.625$ & $10.66\substack{+0.06\\-0.07}$ \\
SDSS1206 & $21.287\substack{+0.222\\-0.184}$ & $32.286\pm5.977$ & $0.110\pm0.020$ & $4.526\pm0.555$ & $0.625$ & $10.78\substack{+0.07\\-0.09}$ \\
HE1104 & $21.274\substack{+0.188\\-0.160}$ & $14.142\pm2.250$ & $0.265\pm0.026$ & $1.033\pm0.034$ & $0.625$ & $11.04\substack{+0.06\\-0.08}$ \\
SDSS0246 & $23.575\substack{+0.394\\-0.289}$ & $3.756\pm1.143$ & $0.381\pm0.098$ & $4.963\pm0.048$ & $0.626$ & $10.69\substack{+0.12\\-0.16}$ \\
HS2209 & $20.385\substack{+0.262\\-0.211}$ & $19.454\pm4.165$ & $1.921\pm1.070$ & $2.953\pm0.438$ & $1.001$ & $11.18\substack{+0.08\\-0.10}$ \\
HE0047 & $22.972\substack{+0.626\\-0.394}$ & $2.496\pm1.093$ & $0.282\pm0.176$ & $3.603\pm1.533$ & $0.625$ & $10.89\substack{+0.16\\-0.25}$ \\
\hline
\end{tabular}
%}
\begin{tablenotes}
      \small
      \item Note: $-$ If any
\end{tablenotes}    
\end{threeparttable}
\end{table*}


\section{Results}
In this section, we present the inference of the stellar mass to derive the scaling relations. We compare our measurements to the ones by the  literature samples to study their evolution.

\subsection{Stellar mass}
[] Color inference for the first three cases to decide their stellar template. 

[] For the other systems, we follow Ding to assume their age based on their redshift.

The uncertainty of the \mstar\ are considered as 0.2 dex, and as studied in~\citet{Ding2017a}, we expect that the uncertainty of the \mbh\ dominate the error budget.

%\section{Black Hole mass estimates}
%We use viral method to estimate the mass of the black hole. The board-line adopted. We aim to adopt the self-consistent to recalibrate the \mbh~of our sample. 
%[] The recipes. [] The board lines and the resulting BH mass and listed in the table. []The summarizing table.
Deriving the the inference of our sample, we over plot our measurements on top of the measurements by \citet{Ding2020}, shown in Figures. The result demonstrate an excellent consistency. and the result support the evolution. 

\subsection{The \mbh-\mstar\ relation}
We combine the \mstar\ to their \mbh\ and compare to the ones in the literature.

\section{Discussion and Conclusion}
[] This work confirm the bright future of lensing.

Look at the future? JWST and TDCOSMO will extend the sample size.


\section*{Acknowledgements}

The Acknowledgements section is not numbered. Here you can thank helpful
colleagues, acknowledge funding agencies, telescopes and facilities used etc.
Try to keep it short.

%%%%%%%%%%%%%%%%%%%%%%%%%%%%%%%%%%%%%%%%%%%%%%%%%%

%%%%%%%%%%%%%%%%%%%% REFERENCES %%%%%%%%%%%%%%%%%%

% The best way to enter references is to use BibTeX:

\bibliographystyle{mnras}
\bibliography{reference} % if your bibtex file is called example.bib


% Alternatively you could enter them by hand, like this:
% This method is tedious and prone to error if you have lots of references
%\begin{thebibliography}{99}
%\bibitem[\protect\citeauthoryear{Author}{2012}]{Author2012}
%Author A.~N., 2013, Journal of Improbable Astronomy, 1, 1
%\bibitem[\protect\citeauthoryear{Others}{2013}]{Others2013}
%Others S., 2012, Journal of Interesting Stuff, 17, 198
%\end{thebibliography}

%%%%%%%%%%%%%%%%%%%%%%%%%%%%%%%%%%%%%%%%%%%%%%%%%%

%%%%%%%%%%%%%%%%% APPENDICES %%%%%%%%%%%%%%%%%%%%%

\appendix

\section{Color inference of the host}
We took the other band data and get the only arc image. We adopt these image and do the SED fitting to get the color information.
\subsection{HE0435}
The HE0435 system has been modelled through three band. In the Figure, we demonstrate the host inference in the image plance for the three bands. 

\subsection{WFI2033}
The WFI2033 system has been modelled through five band.

%%%%%%%%%%%%%%%%%%%%%%%%%%%%%%%%%%%%%%%%%%%%%%%%%%


% Don't change these lines
\bsp	% typesetting comment
\label{lastpage}
\end{document}

% End of mnras_template.tex