% mnras_template.tex 
%
% LaTeX template for creating an MNRAS paper
%
% v3.0 released 14 May 2015
% (version numbers match those of mnras.cls)
%
% Copyright (C) Royal Astronomical Society 2015
% Authors:
% Keith T. Smith (Royal Astronomical Society)

% Change log
%
% v3.0 May 2015
%    Renamed to match the new package name
%    Version number matches mnras.cls
%    A few minor tweaks to wording
% v1.0 September 2013
%    Beta testing only - never publicly released
%    First version: a simple (ish) template for creating an MNRAS paper

%%%%%%%%%%%%%%%%%%%%%%%%%%%%%%%%%%%%%%%%%%%%%%%%%%
% Basic setup. Most papers should leave these options alone.
\documentclass[fleqn,usenatbib]{mnras}

% MNRAS is set in Times font. If you don't have this installed (most LaTeX
% installations will be fine) or prefer the old Computer Modern fonts, comment
% out the following line
\usepackage{newtxtext,newtxmath}
% Depending on your LaTeX fonts installation, you might get better results with one of these:
%\usepackage{mathptmx}
%\usepackage{txfonts}

% Use vector fonts, so it zooms properly in on-screen viewing software
% Don't change these lines unless you know what you are doing
\usepackage[T1]{fontenc}
\usepackage{subfig} %for figure subplot
\usepackage{graphicx}
\usepackage[flushleft]{threeparttable}
\usepackage{xspace}

%%%%%%%%%%%%%%%%%%%%%%%%%%%%%%%%%%%%%%%%%%%%%%%%%%%%%%%%%%%%%%%%%%%
% my commands
\newcommand{\lcdm}{$\Lambda$CDM}
\newcommand{\hst}{{\it HST}}
\newcommand{\hc}{$H_0$}
\newcommand{\efr}{$R_{\mathrm{eff}}$}
\newcommand{\galfit}{{\sc Galfit}}
\newcommand{\mbh}{$\mathcal M_{\rm BH}$}
\newcommand{\lhost}{$L_{\rm host}$}
\newcommand{\mr}{$Mag_{\rm ~R}$}
\newcommand{\halpha}{H$_{\alpha}$}
\newcommand{\Hb}{H$_{\beta}$}
\newcommand{\sersic}{S\'ersic}
\newcommand{\lenstronomy}{{\sc Lenstronomy}}
\newcommand{\reff}{{$R_{\mathrm{eff}}$}}
\newcommand{\kmsMpc}{km~s$^{\rm -1}$~Mpc$^{\rm -1}$}
\newcommand{\kms}{\ifmmode{\,\rm{km}\, \rm{s}^{-1}}\else{$\,$km$\,$s$^{-1}$}\fi}
\newcommand{\sigstar}{{$\sigma_*$}}
\newcommand{\mstar}{{$M_*$}}
\newcommand{\Mgii}{Mg$_{\rm II}$}
\newcommand{\Civ}{C$_{\rm IV}$}
%\newcommand{\farcs}{\mbox{\ensuremath{.\!\!^{\prime\prime}}}}% fractional arcsecond symbol: 0.''0
\newcommand{\sam}{\texttt{SAM}}
\newcommand{\mbii}{\texttt{MBII}}
%%%%%%%%%%%%%%%%%%%%%%%%%%%%%%%%%%%%%%%%%%%%%%%%%%%%%%%%%%%%%%%%%%%
\newcommand{\ding}[1]{\textcolor{red}{[{\bf Xuheng}: #1]}}
\newcommand{\todo}[1]{\textcolor{red}{[{\bf Todo}: #1]}}  
\newcommand{\red}[1]{{ \textcolor{red}{#1}}}
\newcommand{\blue}[1]{{ \textcolor{blue}{#1}}}
\newcommand{\pink}[1]{{\textcolor{magenta}{#1}}}

% Allow "Thomas van Noord" and "Simon de Laguarde" and alike to be sorted by "N" and "L" etc. in the bibliography.
% Write the name in the bibliography as "\VAN{Noord}{Van}{van} Noord, Thomas"
\DeclareRobustCommand{\VAN}[3]{#2}
\let\VANthebibliography\thebibliography
\def\thebibliography{\DeclareRobustCommand{\VAN}[3]{##3}\VANthebibliography}


%%%%% AUTHORS - PLACE YOUR OWN PACKAGES HERE %%%%%

% Only include extra packages if you really need them. Common packages are:
\usepackage{graphicx}	% Including figure files
\usepackage{amsmath}	% Advanced maths commands
\usepackage{amssymb}	% Extra maths symbols

%%%%%%%%%%%%%%%%%%%%%%%%%%%%%%%%%%%%%%%%%%%%%%%%%%

%%%%% AUTHORS - PLACE YOUR OWN COMMANDS HERE %%%%%

% Please keep new commands to a minimum, and use \newcommand not \def to avoid
% overwriting existing commands. Example:
%\newcommand{\pcm}{\,cm$^{-2}$}	% per cm-squared

%%%%%%%%%%%%%%%%%%%%%%%%%%%%%%%%%%%%%%%%%%%%%%%%%%

%%%%%%%%%%%%%%%%%%% TITLE PAGE %%%%%%%%%%%%%%%%%%%

% Title of the paper, and the short title which is used in the headers.
% Keep the title short and informative.
\title[Mass relations by lensed AGN hosts]{Testing the Evolution of the Correlations between Supermassive Black Holes and their Host Galaxies using Eight Strongly Lensed quasars}

% The list of authors, and the short list which is used in the headers.
% If you need two or more lines of authors, add an extra line using \newauthor
\author[X. Ding et al.]{
Xuheng Ding,$^{1}$\thanks{E-mail: dxh@astro.ucla.edu}
Tommaso Treu,$^{1}$
Simon Birrer,$^{2}$
Adriano Agnello,$^{3}$\newauthor
Dominique Sluse,$^{4}$
Chris Fassnacht,$^{5}$
Matthew W. Auger,$^{6}$
Kenneth C. Wong,$^{7}$ \newauthor
Sherry H. Suyu,$^{8,9,10}$
Takahiro Morishita,$^{11}$
\pink{
Cristian E. Rusu,$^{12,13,14}$
Aymeric Galan,$^{15}$}
\\
\\
% List of institutions
$^{1}$Department of Physics and Astronomy, University of California, Los Angeles, CA, 90095-1547, USA\\
$^{2}$Kavli Institute for Particle Astrophysics and Cosmology and Department of Physics, Stanford University, Stanford, CA 94305, USA\\
$^{3}$European Southern Observatory, Karl-Schwarzschild-Strasse 2, 85748 Garching, Germany\\
$^{4}$STAR Institute, Quartier Agora - All\'ee du six Ao\^ut, 19c B-4000 Li\`ege, Belgium\\
$^{5}$Department of Physics, University of California, Davis, CA 95616, USA\\
$^{6}$Institute of Astronomy, University of Cambridge, Madingley Road, Cambridge CB3 0HA, UK\\
$^{7}$Kavli IPMU (WPI), UTIAS, The University of Tokyo, Kashiwa, Chiba 277-8583, Japan\\
$^{8}$Max-Planck-Institut f{\"u}r Astrophysik, Karl-Schwarzschild-Str.~1, 85748 Garching, Germany\\
$^{9}$Physik-Department, Technische Universit\"at M\"unchen, James-Franck-Stra\ss{}e~1, 85748 Garching, Germany\\
$^{10}$Academia Sinica Institute of Astronomy and Astrophysics (ASIAA), 11F of ASMAB, No.1, Section 4, Roosevelt Road, Taipei 10617, Taiwan\\
$^{11}$Space Telescope Science Institute, 3700 San Martin Drive, Baltimore, MD 21218, USA\\
$^{12}$National Astronomical Observatory of Japan, 2-21-1 Osawa, Mitaka, Tokyo 181-8588, Japan\\
$^{13}$Subaru Telescope, National Astronomical Observatory of Japan, 650 N Aohoku Pl, Hilo, HI 96720\\
$^{14}$Department of Physics, University of California, Davis, 1 Shields Avenue, Davis, CA 95616, USA\\
$^{15}$Institute of Physics, Laboratory of Astrophysics, Ecole Polytechnique 
F\'ed\'erale de Lausanne (EPFL), Observatoire de Sauverny, 1290 Versoix, 
Switzerland
}

% These dates will be filled out by the publisher
\date{Accepted XXX. Received YYY; in original form ZZZ}

% Enter the current year, for the copyright statements etc.
\pubyear{2020}

% Don't change these lines
\begin{document}
\label{firstpage}
\pagerange{\pageref{firstpage}--\pageref{lastpage}}
\maketitle

% Abstract of the paper
\begin{abstract}
  One of the main challenges in using high redshift active galactic nuclei to study the correlations between the mass of the supermassive Black Hole (\mbh) and the properties of their active host galaxies is instrumental resolution. Strong lensing magnification effectively increases instrumental resolution and thus helps to address this challenge. In this work, we study eight strongly lensed active galactic nuclei (AGN) with deep {\it Hubble Space Telescope}(\hst) imaging, using the \pink{lens modelling code} \lenstronomy\ to reconstruct the image of the source. Using the reconstructed brightness of the host galaxy, we infer the host galaxy stellar mass based on stellar population models. 
  \mbh\ are estimated from broad emission lines using standard methods. Our results are in good agreement with recent work based on non-lensed AGN, providing additional evidence that the correlation evolves over cosmic time.  At the moment, the sample size of lensed AGN is small and thus they provide mostly a consistency check on systematic errors related to resolution for the non-lensed AGN. However, the number of known lensed AGN is expected to increase dramatically in the next few years, through dedicated searches in ground and space based wide field surveys, and they may become a key diagnostic of black hole and galaxy co-evolution. 
\end{abstract}

% Select between one and six entries from the list of approved keywords.
% Don't make up new ones.
\begin{keywords}
galaxies: evolution -- galaxies: active -- gravitational lensing: strong
\end{keywords}

%%%%%%%%%%%%%%%%%%%%%%%%%%%%%%%%%%%%%%%%%%%%%%%%%%

%%%%%%%%%%%%%%%%% BODY OF PAPER %%%%%%%%%%%%%%%%%%
\section{Introduction}
The tight correlations between the masses (\mbh) of supermassive black holes (BHs) and their host galaxies properties including stellar mass (\mstar), stellar velocity dispersion ($\sigma_*$) and luminosity (\lhost), known as scaling relations, are usually considered as a result of their co-evolution ~\citep[e.g.,][]{Mag++98, F+M00, Geb++01b, M+H03, Gul++09,Beifi2012, H+R04, Gra++2011}. Cosmological simulations of structure formation are able to reproduce the local tight relation based on physical mechanism by invoking active galactic nucleus (AGN) feedback as the physical connection~\citep{Springel2005, Hopkins2008, Matteo2008, DeG++15} or having them share a common gas supply~\citep{Cen2015, Menci2016}. However, it has also been suggested that the correlations arise statistically, without any physical coupling, as a result of stochastic mergers~\citep{Peng2007, Jahnke2011, Hirschmann2010}. 

A powerful way to understand the origin of the correlations is to study them as a function of redshift, determining how and when they emerge and evolve over cosmic time~\citep[e.g.,][]{TMB04,Sal++06,Woo++06, Jah++09,SS13,Sun2015, Park15}. Recently, based on a sample of 32 X-ray-selected type-1 AGNs in deep survey fields, \citet{Ding2020a} (hereafter, D20) measured the scaling relations in the redshift range $1.2<z<1.7$ using multi-color Wide Field Camera 3 (WFC3) \hst\ imaging. Combining the new sample with published samples in both local and intermediate \pink{(i.e., $0.35<z\lesssim1.2$)} redshift ranges, D20 strengthen the support for an evolution scenario in which the growth of BHs evolution predates that of  the bulge component of the host galaxy. In a follow-up paper, \citet{Ding2020b} compared the D20 measurements to the predictions by the numerical simulations including the hydrodynamic simulation MassiveBlackII~\citep{Khandai2015} and a semi-analytic model~\citep{Menci2014}. The observed tightness of the  scaling relations at high redshift is consistent with the hypothesis that AGN feedback drives the scaling correlations, and disfavors the hypothesis of the correlations being purely stochastic in nature. 
% , a larger sample at the higher redshift range would be useful to increase the resolution of the evolution and confirm these hypothesises.

As the samples of AGN grow to increase statistical power, it is very important to make sure that systematics do not dominate the error budget. One of the main potential source of systematics is the finite resolution of \hst\ images. Even with \pink{the modern} techniques, AGN hosts remain barely resolved at \hst\ resolution, and therefore it would be very useful to verify the results at higher resolution. This is the goal of this work.

It has been long recognized that lensed AGNs, by virtue of magnification, can provide unique insights into the scaling relations the distant universe~\citep{Peng2006}, provided that the lens modelling could be accurately derived. Aiming at verifying the fidelity of the modern lens modelling technique, \citet{Ding2017a} carried out extensive and realistic simulations tests based on the deep \hst\ observations for a sample of eight lens systems in the \hc\ Lenses in COSMOGRAIL's Wellspring \citep[H0LiCOW\footnote{\url{http://www.h0licow.org/}},][]{Suyu2017} collaboration. They confirm that the reconstruction of the lensed host galaxy properties can be recovered with better precision and accuracy than the typical \mbh\ uncertainty. Then, \citet{Ding2017b} applied the advanced techniques to two strongly lensed systems analyzed by the H0LiCOW collaboration~\citep{Suyu2013, Wong2017} to study their \mbh-\lhost\ relations and obtained consistent scaling relations compared with the samples from the literature.

In this work, we expand the measurements of the \mbh-\mstar\ relation using the full sample of eight lensed AGN introduced by \citet{Ding2017a}.  In order to take advantage of the excellent quality of the data, we develop an independent approach to achieve a one-step inference of the host galaxy photometry from the eight lensed AGNs. We adopt a set of extended modelling choices to \pink{estimate the host property uncertainty level and} ensure the accuracy of our measurements. We compare the inference of our new measurements to the ones that have been modelled by the H0LiCOW collaboration to make the cross-check. To obtain an accurate inference of stellar mass, we utilize the multi-band imaging data taken with the \hst\ to obtain the color information for 3/8 of our targets. \pink{We assume a typical stellar population for the rest 5/8 of the sample.} Furthermore, we adopt a class of self-consistent recipes to recalibrate the \mbh\ of our sample, in a manner consistent with D20. Given the similar redshift range, the high data quality, and consistent state-of-the-art techniques, this sample of lensed AGN provides an excellent verification of the D20 results.

This paper is structured as follows. In Section~\ref{sec:sample_select}, we introduce the sample including their imaging data and the BH masses. In Section~\ref{sec:lens_model}, we describe our approach designed to infer the lens models and reconstruct the host galaxy. We use the inferred photometry to derive the stellar mass and the scaling relations and combine with D20 sample to study the evolution in Section~\ref{sec:result}. Discussion and conclusions are drawn in Sections~\ref{sec:diss} and~\ref{sec:con}. Throughout this paper, we adopt a standard concordance cosmology with parameters adopted as $H_0= 70$ km s$^{-1}$ Mpc$^{-1}$, $\Omega{_m} = 0.30$, and $\Omega{_\Lambda} = 0.70$, \pink{to compute the luminosity distance and estimate the host absolute brightness}. Magnitudes are presented in the AB system. A Chabrier initial mass function (IMF) is employed for the sample, to be consistent with D20.

\section{Sample Selection and Black Hole Mass Estimates}\label{sec:sample_select}
We adopt eight lens systems from the H0LiCOW collaboration including HE0435$-$1223, RXJ1131$-$1231, WFI2033$-$4723, HE1104$-$1805, SDSS1206$+$4332, SDSS0246$-$0825, HE0047$-$1756 and HS2209$+$1914. We refer to~\citet{Suyu2017, Ding2017a} for the descriptions of these data. For conciseness, in the rest of this paper, we abbreviate each lens name to four digits (e.g., HE0435$-$1223 to HE0435). 

Based on the observational data of these eight systems, \citet{Ding2017a} performed an extensive and realistic simulation exercise using the \hst\ image data and confirmed that the source reconstruction using the lens modelling technique is trustworthy. We summarize the information for the eight systems, including \pink{their redshift, data conditions and the references} in Table~\ref{data_set}.
Besides the imaging data shown in this table, we also analyzed the multi-band \hst\ imaging data to derive the host color information for 3/8 systems. As we show in Section~\ref{sec:mstar}, we use the color information to fit for the best stellar population to improve the accuracy of the estimate of \mstar. 

The sample of lensed AGN is too limited in size to constrain evolution by itself. Therefore, we use it primarily to verify the results of D20.  The D20 sample includes 32 AGN measurements in the redshift range $1.2<z<1.9$. \pink{They also collected 59 intermediate redshift (i.e., $0.35<z\lesssim1.2$) AGN measurements~\citep{Bennert11, SS13, Cisternas2011} and 55 local (i.e., $z\lesssim0.007$) measurements~\citep{Bennert++2011, H+R04}.} It is worth noting that they are so far the largest \hst\ imaging AGN sample with redshift range up to $z\sim1.9$. 
 
\begin{table*}
\centering
%\begin{threeparttable}
\caption{Summary of lensed AGN observational details.}\label{data_set}
\resizebox{15cm}{!}{
     \begin{tabular}{ccccccccc}
        \hline
Object ID & $z_s$ & camera & Filter & exposure & Program ID & PI & pixel scale & References \\
 & & & & time (s) &&& (drizzled) & \\ \hline
HE0435$-$1223 & 1.69 & WFC3-IR & F160W & 9340 & 12889 & S. H. Suyu & $0\farcs{}08$ & (1), (2)\\
RXJ1131$-$1231 & 0.65 & ACS & F814W & 1980 & 9744 & C.S. Kochanek & $0\farcs{}05$ &(3), (4)\\
WFI2033$-$4723 & 1.66 & WFC3-IR & F160W & 26257 & 12889 & S. H. Suyu & $0\farcs{}08$ & (5), (2) \\
SDSS1206$+$4332 & 1.79 & WFC3-IR & F160W & 8457 & 14254 & T. Treu & $0\farcs{}08$ & (6), (7)\\
HE1104$-$1805 & 2.32 & WFC3-IR & F160W & 14698 & 12889 & S. H. Suyu  & $0\farcs{}08$ & (8), (9)\\
SDSS0246$-$0825 & 1.68 & WFC3-UVIS & F814W & 8481 & 14254 & T. Treu & $0\farcs{}03$& (10), (11) \\
HS2209$+$1914$^a$  & 1.07 & WFC3-UVIS & F814W & 9696 + 4542 & 14254 & T. Treu & $0\farcs{}03$ & (12), (13) \\
HE0047$-$1756 & 1.66 & WFC3-UVIS & F814W & 9712 & 14254 & T. Treu & $0\farcs{}03$ & (14), (15) \\
        \hline
     \end{tabular}
    }
\begin{tablenotes}
      \small
      \item Note: $-$ The observational information is also given by~\citet{Ding2017a}.
      \item $^a$: HS2209, was visited by \hst\ twice ({\it vis05} and {\it vis06}) at different orientations. The exposure time is thus given separately.
      \item References:$-$ (1) \citet{HE0435_discover}; (2) \citet{Sluse2012}; (3) \citet{1131_discover}; (4) \citet{1131_redshift}; (5) \citet{2033_discover}; (6) \citet{1206_discover}; (7) \citet{Eul++13}; (8) \citet{HE1104_discover}; (9) \citet{1104_d_redshift}; (10) \citet{0246_discover}; (11) \citet{0246_redshift}; (12) \citet{2209_discover};  (13) \citet{2209_more}; (14) \citet{0047_discover}; (15) \citet{Ofe++06};
    
\end{tablenotes}  
%\end{threeparttable}
\end{table*}

To ensure the consistency of \mbh\ estimates based on different broad lines, we adopt the following set of self-consistent recipes following D20:
\begin{eqnarray}
\label{recipe}
\log \left(\frac{\mathcal M_{\rm BH}}{M_{\odot}}\right)&~=~& a+b \log \left(\frac{ \rm L _{\lambda_{line}}}{10^{44}{\rm erg~s^{-1}}}\right) \nonumber\\
&~+~& 2 \log \left(\frac{\rm FWHM(line)}{1000 ~{\rm km~s^{-1}}}\right) , 
\end {eqnarray}
\pink{where $\lambda_{\rm line}$ is the reference wavelength of the local continuum luminosities for different emission lines. The following values are adopted:}
%
$a$\{\Civ, \Mgii, \Hb\}=\{6.322, 6.623, 6.910\},
$b$\{\Civ, \Mgii, \Hb\}=\{0.53, 0.47, 0.50\},
$\lambda_{\rm line}$\{\Civ, \Mgii, \Hb\}=\{1350, 3000, 5100\} ($\AA$).
%
The broad line properties of our samples are adopted from the literature, with a few corrections/exceptions. % \citep{Sluse2012, Peng2006, Shen2011}.
The line properties of SDSS1206 have been inferred by~\citet{Shen2011}. However, this system was investigated as a non-lensed AGN and the lensing magnification on the intrinsic continuum luminosity was not considered. We follow~\citet{Birrer2019} and apply the same magnification correction on the $\log(L_\lambda)$ in this work. For the rest of the lens sample except HS2209, their broad line properties are adopted from the literature~\citep{Sluse2012, Peng2006} taking into account lensing magnification. For HS2209, the spectrum was observed at the Keck-II Telescope in September 2015. We derived the line properties from the spectrum using the same approach as~\citet{Sluse2012}, \pink{based on \Mgii}. The \mbh\ measurements, together with the properties of the board-line, are listed in Table~\ref{mbh}. 
The uncertainty of the \mbh\ are estimated to be $0.4$~dex.


\begin{table}
\centering
%\begin{threeparttable}
\caption{Summary of \mbh\ estimates, based on equation~\eqref{recipe}.}\label{mbh}
\resizebox{8.5cm}{!}{
     \begin{tabular}{cccccc}
        \hline
Object ID & Line(s) & FWHM & log($L_\lambda$) & $\log$\mbh & ref.  \\
& & (\kms) & (${\rm erg~s^{-1}}$) & (M$_{\odot}$) \\ \hline
HE0435 & \Mgii & 4930 & 45.14 & 8.54 & (1) \\
RXJ1131 & \Mgii/\Hb & 5630/4545 & 44.29/44.02 & 8.26/8.23 & (1)\\
WFI2033 & \Mgii & 3960 & 45.19 & 8.38 & (1) \\
SDSS1206 & \Mgii & 5632 & 45.01 & 8.77 & (2) \\
HE1104 & \Civ & 6004 & 46.18 & 9.03 & (3) \\
SDSS0246 & \Mgii & 3700 & 45.19 & 8.32 & (1) \\
HS2209 & \Mgii & 3245 & 45.71 & 8.75 & here  \\
HE0047 & \Mgii & 4145 & 45.59 & 8.60 & (1) \\
        \hline
     \end{tabular}
    }
\begin{tablenotes}
\small
\item Note: $-$ The broad line properties.\\ %For HS2209, the properties are derived in this work using the same approach as~\citet{Sluse2012}.
Reference: (1)~\citet{Sluse2012}, (2)~\citet{Shen2011}, (3)~\citet{Peng2006}.\\
%$^a$: For SDSS1206, we follow~\citet{Birrer2019} and correct the AGN continuum luminosity $\log(L_\lambda$) by taking account the magnification effect {\bf [TT: how about the others?].}
\end{tablenotes}  
%\end{threeparttable}
\end{table}

\section{surface photometry inference}\label{sec:lens_model}
%
In this section, we describe how we derived surface photometry of the lensed galaxies taking lensing effects into account. Lens models of 4/8 systems, namely HE0435~\citep{Wong2017}, RXJ1131~\citep{Suyu2013}, WFI2033~\citep{Rusu2019} and SDSS1206~\citep{Birrer2019} have been published by the H0LiCOW project. The goal of those models was strong lens time-delay cosmography~\citep{Refsdal1966, Treu2016}, and the reconstruction of the host galaxy light \pink{(via pixellated distribution~\citep{Suy++06} or shapelets~\citep{Refregier2003})} was a byproduct. \citet{Ding2017b} used those reconstructions for two of the systems (HE0435 and RXJ1131), and then fitted a simply parametrized surface brightness profile to the reconstruction to \pink{measure the host properties as the non-lensed AGN case}.
% mimic what is done in non-lensed AGN.

In this work, in order to reproduce more closely and uniformly what is
done in non-lensed AGN, we develop a strategy to obtain a one step
measurement of the host galaxy light described by a \sersic\ surface
brightness profile. For the 4/8 systems already studied by the H0LiCOW
project, we will make a comparison to characterize systematic
uncertainties related to the modelling techniques.

\subsection{Data Preparation and Modelling Setup}\label{sec:Modelling}

We follow the standard procedure as described by D20 to prepare the fitting ingredients including the lensing imaging data, noise level map and the PSF information. The imaging data are first drizzled to a higher resolution with a Gaussian kernel; the adopted resolutions are listed in Table~\ref{data_set}. We then adopt the {\sc Photutils}~\citep{photutils} Python package to model the global background light in 2-D, based on the \texttt{SExtractor} algorithm. We remove the background light and cut the clear image data into postage stamp at a suitable size. We draw the image mask for each system to define the region in which the pixels will be used to calculate the likelihood; see the top-\pink{middle} panel in each subplot in Figure~\ref{fig:image_inference}.

We carry out a forward modelling process to simultaneously constrain the lens model, subtract the central AGN light, and infer the photometry of the host galaxy. For any extended objects, including the lensing galaxy and source galaxy, we assume their surface brightness can be described by the 2-D elliptical \sersic\ profile. We start with a single \sersic\ profile and consider to use two \sersic\ if any significant residual indicate multiple components. For a single \sersic\ profile, we set the prior of the \sersic\ index value $n$ between $[0.5-5.0]$ to avoid unphysical results\footnote{It has been shown in~\citet{Ding2017a} that choosing this prior for \sersic\ index $n$ yields unbiased host magnitude inferences.}. The bright nucleus is unresolved and modelled by a scaled point spread function (PSF) in the image plane. We also impose that the AGN and its host galaxy have the same center. Following standard practice that has been shown to produce good models \citep[e.g.,][and references therein]{Treu2010}, we adopt elliptical power-law density profiles to define the surface mass density of the deflector, with an external shear.

We employ the imaging modelling tool \lenstronomy\footnote{\url{https://github.com/sibirrer/lenstronomy}}~\citep{lenstronomy} to perform the fitting task. Inspired by D20, we adopt a set of extensive modelling choices to perform the fitting \pink{for each system, repeatedly. Then, based on the fittings using different choices,} we apply a statistical measurement and perform a weighting algorithm to derive the final inference and estimate the uncertainty \pink{level}. The modelling choices that we consider include the follows.
\begin{enumerate}
\item Following common practice, we select all the bright, isolated stars across the image frame of targets to define the PSF. Each selected star is considered as an initial PSF guess for the fit.
\item The central pixels of the AGNs are very bright and can be affected by large systematic errors during the interpolation of the subsampled PSF. To avoid overfitting the noise, we adopt two different modelling options: 1)~manually boost the noise estimate in the central area to effectively infinite ({\it noise boost}); 2)~perform the iterative PSF estimation as introduced by~\citet{Chen2016, Birrer2019} ({\it PSF iteration}).
\item To calculate the ray tracing under a higher resolution grid, we choose to oversample the model by a factor  $2\times2$ or $3\times3$ pixel$^{-1}$.
\item Using the lensing imaging alone, it is difficult to constrain the power-law slope, \pink{especially the model of our host surface brightness is much simplified}. To mitigate the overfitting of the slope value, we repeat the fit for three values $[1.9, 2.0, 2.1]$ that are meant to bracket the range observed in lens galaxies \citep[e.g.,][]{Auger2010}.
\end{enumerate}

In general, for one lens system with a number of $N$ initial PSF guesses, we perform in total $N\times12$ \pink{(i.e., $2$ by (ii) $\times~2$ by (iii) $\times~3$ by (iv)) fits}. After all fits are completed, we rank their performance based on their best-fitted $\chi^2$ value. Since there is no evidence of a better performance for the options between {\it noise boost} and {\it PSF iteration}, we selected the top-4 fittings from each of them and then combine their best-fit results based on the weighting algorithm introduced by D20. The weights are defined by:
\begin{eqnarray}
\label{eq:weights}
w_i = {\rm exp} \big(- \alpha \frac{ (\chi_i ^2 - \chi_{best} ^2 )}{2 \chi_{best} ^2} \big),
\end{eqnarray} 
where the $\alpha$ is an inflation parameter\footnote{Defining $\alpha$ as the {\it inflation} parameter it is assumed that it is larger than ~$1$, so as to err on the side of caution and include more choices that would be allowed strictly by statistical noise considerations.} so that when $i=4$:
\begin{eqnarray}
\label{eq:alpha}
\alpha \frac{ \chi_{i=4} ^2 - \chi_{best} ^2 }{2 \chi_{best} ^2} = 2.
\end{eqnarray} 
Then, the results for {\it noise boost} and {\it PSF iteration} are combined equally to derive the value of host properties and the root-mean-square error (i.e., $\sigma$ level) based on the weights of the $4+4$ (i.e., {\it noise boost} and {\it PSF iteration}) options by:
\begin{eqnarray}
\label{eq:infer_value}
\bar{x}  =  \frac{  \sum_{i=1}^{N}   x_i * w_i  }{\Sigma w_i} ,
\end{eqnarray} 
\begin{eqnarray}
\label{eq:infer_scatter}
\sigma =   \sqrt{ \frac{  \sum_{i=1}^{N}   (x_i -  \bar{x} ) ^2 * w_i  }{\Sigma w_i} }.
\end{eqnarray} 

Our approach uses the relative goodness of fit and ensures that at least 8 sets of best fits are used to estimate the range of systematic uncertainties. \pink{Note that the lens mass slope is fixed in each fitting, and the statistical uncertainty is much smaller than the systematic one.} 

\subsection{Photometry Inference}\label{sec:photometry}
In this subsection, we describe the details of the fitting for each system and present the inference of the photometry of the host galaxy.

\subsubsection{HE0435}\label{subsec:HE0435}
We follow the approach described in the previous section. We select 5 isolated stars in this field as initial PSFs to input to the fitting, for a total of 60 fits. Based on the top-eight choices, we perform the weighting algorithm and measure the host flux, host-to-total flux ratio, effective radius and \sersic\ index. The inference results are shown in Table~\ref{tab:host_measure}, (2)$-$(6) columns.

The inference by the best-fit lens model for HE0435 is shown in Figure~\ref{fig:image_inference}-(a). Not surprisingly, we note that the residuals level in the normalized plot appears to be larger than the ones presented by~\citet{Wong2017}. This is primarily due to the fact that the surface brightness of the host galaxy in our model is defined as a \sersic\ profile, which is relatively simple and smooth compared to the pixellated reconstruction technique adopted by {\sc Glee}~\citep{Suy++06, Halk2008, S+H10, Suyu2012} or the shapelet technique adopted by \lenstronomy~\citep{Birrer2015}. The smooth features of \sersic\ profile can not capture the clumps in the host galaxy, such as the star-forming regions. Nevertheless, our approach is sufficient to derive a self-consistent one-step inference of the global host light in terms of the \sersic\ flux, i.e. the quantity commonly measured in the literature for non-lensed samples. 

As a cross-check, we compare our host magnitude to the inference of HE0435 by~\citet{Ding2017b}. They inferred the \sersic\ magnitude by fitting to the pixelized host galaxy as reconstructed by~\citet{Wong2017}. The host magnitude measured by~\citet{Ding2017b} is: $mag = 21.75 \pm 0.13$. %, $R_{eff} = 0.82 \pm 0.14$ and $n = 3.94 \pm 0.14$. 
The results are in excellent agreement with the measurement reported here, i.e. $mag = 21.50 \pm 0.35$.

\subsubsection{RXJ1131}
A lens model of RXJ1131 based on ACS/F814W data has been presented by~\citet{Suyu2013}. The host galaxy of this system is lensed to an extended arc. A clear bulge and disk component can be identified. Thus, we describe the host galaxy with two \sersic\ profiles with index values $n$ fixed to $1$ and $4$, respectively, to mimic the light distribution of the disk and bulge. In addition, following~\citet{Suyu2013}, we consider the perturbations by the small object ($0\farcs{5}$ in the north) and describe it as a SIS and \sersic\ for its mass and light, respectively.

The photometry of the RXJ1131 host galaxy is fitted with a set of 4 initial PSFs. The results are summarized in Table~\ref{tab:host_measure}, with the best-fit result shown in Figure~\ref{fig:image_inference}-(b).

We also compare our measurement to the previous reconstructions by \citet{Ding2017b}. Based on the reconstructions by~\citet{Suyu2013}, the inference of the host magnitudes by~\citet{Ding2017b} are $mag_{\rm bulge} = 21.81 \pm 0.28$ and $mag_{\rm disk} = 20.07\pm0.06$. The results are in excellent agreement with our inferred bulge magnitude ($21.80\pm0.21$)   shown in Table~\ref{tab:host_measure}. However, our inferred disk magnitude ($19.33\pm0.16$) is brighter than that reported by \citet{Ding2017b}. This difference is not surprising because our reconstruction of the host galaxy is based on a much more extended region to collect the disk light, compared with~\citet{Suyu2013} who performed the lens modelling using a smaller lens mask (see Figure~4 therein). \pink{Note that \citet{Ding2017b} used a \sersic\ to model the reconstructions by~\citet{Suyu2013}, and then the inferred host surface brightness is computed by collecting the \sersic\ light within the aperture where host was reconstructed. This aperture in the source plane is decided by the mask region in the image plane, thus a smaller mask region could lead the inference of the host brightness fainter. For a consistency check, we find that our inferred effective radius is very consistent, $0\farcs{}90 \pm 0\farcs{}06$ (here) and $0\farcs{}84 \pm 0\farcs09$~\citep{Ding2017b}.} The difference between the \citet{Suyu2013} reconstruction and the one presented here makes sense in terms of the different goals of the two studies. While our primary aim is to reconstruct the host galaxy photometry, for \citet{Suyu2013} it was only a byproduct on the way to time-delay cosmography.

\subsubsection{WFI2033}
WFI2033 is the last quadruply lensed system in our sample whose lens model has been previously investigated \citep{Rusu2019}. There is a satellite galaxy in the north. However, as noted by~\citet{Rusu2019}, the satellite galaxy has much smaller mass than the main deflector, thus we ignore its influence on the magnification but only fit its light using a \sersic\ model. We select a total of 8 initial PSF stars to model this system.

The final inference results are presented in Table~\ref{tab:host_measure} and Figure~\ref{fig:image_inference}-(c). We compare our inference to the previously reconstructed host galaxies. Modelling the reconstructed host by~\citet{Rusu2019}, (i.e., the Figure~4 bottom-right plane therein) as a \sersic\ profile, we infer the $mag = 21.98 \pm 0.15$, which is very consistent to our inference (i.e., $mag = 21.78 \pm 0.25$).

\subsubsection{SDSS1206}
SDSS1206 is a unique system -- the AGN is doubly imaged by the deflector while most of the host falls inside the inner caustic and ends up being quadruply imaged. Following~\citet{Birrer2019}, we consider the galaxy triplet group at the north-west and use a single SIS model to denote their overall mass perturbation. Moreover, as noted by~\citet{Birrer2019}, a sub-clump is located in the north which is hardly visible \citep[see Figure~1 in][]{Birrer2019}. We model this sub-clump as a SIS mass model and a circular \sersic\ light model with joint centroids. It is worth noting that we are using the same imaging modelling tool as~\citet{Birrer2019}.

Due to the limited number of stars in the field of view, there are only 2 stars available as initial PSF. To expand the volume of modelling options, we also take the stack of the two bright stars as derived by~\citet{Birrer2019}, as a third initial PSF. We find a visible residuals at the fitted lensed arcs region using a single \sersic\ model as host. However, a double \sersic\ model does not significantly improve the goodness of fit. Thus, we adopt the single \sersic\ model in our final inference. Our inferred results are presented in Table~\ref{tab:host_measure} and Figure~\ref{fig:image_inference}-(d).


\subsubsection{HE1104}
HE1104 is a typical doubly imaged quasar. We have selected in total of 5 initial PSFs to perform the fit. There is an object at the northeast. However, since we do not know its redshift and considering it is further away from the lens we do not model it explicitly but just mask it out in the fitting.

The inference is presented in Table~\ref{tab:host_measure} and Figure~\ref{fig:image_inference}-(e). the lensed arcs can be clearly seen from the bottom-left panel, indicating that the host galaxy is well detected.

\pink{It is worth noting that the HE1104 has been modelled by~\citet{Peng2006} based on the \hst/NICMOS H-band (F160W) imaging data. Their inferred host light is $20.14\pm0.30$ mag in Vega system, which is also consistent to our inference ($20.00\pm0.15$ mag in Vega).}
%zeropoint 25.9463(AB) - 24.6949(Vega) = 1.2514
%HE1104 mag:  21.25 (AB)  --> 20.00 (Vega)

\subsubsection{SDSS0246}
Having been imaged with WFC3-UVIS/F814W, the resolution of the data for this system (together with the remaining two systems) is much higher than for those imaged in the IR, with a drizzled pixel scale of $0\farcs{}03$. However, the arcs are much fainter compared with those of other systems imaged in the IR band. As a result, fewer pixels with signal are available for the fit than for other systems. Nevertheless, the host inference is successfully reconstructed as shown in Figure~\ref{fig:image_inference}-(f) and Table~\ref{tab:host_measure} (3 initial PSF guesses were adopted).  

\subsubsection{HS2209}
HS2209 was imaged by \hst\ during two visits ($vis05$ and $vis06$) at different orientations. We modelled the two visits separately and recovered mutually consistent host galaxy magnitudes. However, we found that the data from {\it vis06} can be modelled with smaller residuals; thus the inference based on {\it vis06} was adopted as our best estimate (using 7 initial PSF stars), as listed in Table~\ref{tab:host_measure}. The inference for this system is summarized in Figure~\ref{fig:image_inference}-(g).

\subsubsection{HE0047}
HE0047 is the most challenging system in our sample with the lowest SNR of the lensed arcs. The results, based on 3 initial PSFs, are summarized in Figure~\ref{fig:image_inference}-(f) and Table~\ref{tab:host_measure}. The host magnitude of the HE0047 system has a relatively large uncertainty as reflected in the error bars.

\begin{figure*}
\centering
\begin{tabular}{c c}
\subfloat[HE0435]{\includegraphics[trim = 40mm 25mm 40mm 25mm, clip, width=0.5\textwidth]{fig/HE0435_inference.png}}&
\subfloat[RXJ1131]{\includegraphics[trim = 40mm 25mm 40mm 25mm, clip, width=0.5\textwidth]{fig/RXJ1131_inference.png}}\\
\subfloat[WFI2033]{\includegraphics[trim = 40mm 25mm 40mm 25mm, clip, width=0.5\textwidth]{fig/WFI2033_inference.png}}&
\subfloat[SDSS1206]{\includegraphics[trim = 40mm 25mm 40mm 25mm, clip, width=0.5\textwidth]{fig/SDSS1206_inference.png}}\\
\subfloat[HE1104]{\includegraphics[trim = 40mm 25mm 40mm 25mm, clip, width=0.5\textwidth]{fig/HE1104_inference.png}}&
\subfloat[SDSS0246]{\includegraphics[trim = 40mm 25mm 40mm 25mm, clip, width=0.5\textwidth]{fig/SDSS0246_inference.png}}\\
\subfloat[HS2209]{\includegraphics[trim = 40mm 25mm 40mm 25mm, clip, width=0.5\textwidth]{fig/HS2209_inference.png}}&
\subfloat[HE0047]{\includegraphics[trim = 40mm 25mm 40mm 25mm, clip, width=0.5\textwidth]{fig/HE0047_inference.png}}\\
\end{tabular}
\caption{\label{fig:image_inference} 
Illustrations of the inference using the best-fit lens model for each system using the AGN center noise level boosted approach. For each figure, the panels from left to right are the follows: Top row: (left to right): (1) observed data, (2) best-fit model (3) residuals divided by the uncertainty level. \pink{Some of the residual looks clumpy; however, they do not affect the host property measurements, for more details see Section~\ref{sec:diss}} ; Bottom row: (left to right): (4) data minus the model PSF and deflector  (i.e., pure lensed arc image), (5) the model of the lensed arc, (6) reconstructed host galaxy in the source plane with caustic line drawn as yellow line. Across all the systems, the lensed arc \pink{feature} can be clearly seen in the \pink{fourth} panel, indicating strong evidence of a detection. In panel (3) and (4), we use white regions to indicate the area where the noise level is boosted. 
}
\end{figure*} 

\begin{table*}
\renewcommand{\arraystretch}{1.25}
%\setlength{\tabcolsep}{20pt}
\centering
  \begin{threeparttable}
\caption{Summary of lensed AGN host inference.}\label{tab:host_measure}    
%\resizebox{12cm}{!}{
     \begin{tabular}{cccccccc}
\hline     
(1) & (2) & (3) & (4) & (5) & (6) & (7) & (8) \\    
Object ID & intrinsic magnitude & magnitude & Host Flux Ratio (to Total) & \reff\ & \sersic\ $n$ & stellar population & $\log (M_{*}$)  \\
 & (source plane) & (image plane) & ( $\%$, \pink{Total = Host + AGN}) & (arcsec) & & age (Gyr) & (M$_{\odot}$) \\ \hline
HE0435 & $21.49\substack{+0.40\\-0.29}$ & $18.58\substack{+0.30\\-0.23}$ & $36.0\pm11.1$ & $0.28\pm0.02$ & $2.71\pm0.20$ & $1.50$ & $10.91\substack{+0.12\\-0.16}$ \\
RXJ1131$_{\rm bulge}$ & $21.80\substack{+0.23\\-0.19}$ & $18.70\substack{+0.07\\-0.06}$ & $7.1\pm1.4$ & $0.13\pm0.02$ & fix to 4 & $3.00$ & $10.39\substack{+0.08\\-0.09}$ \\
RXJ1131$_{\rm disk}$ & $19.33\substack{+0.17\\-0.15}$ & $17.14\substack{+0.08\\-0.07}$ & $69.2\pm10.1$ & $0.90\pm0.06$ & fix to 1 & $1.50$ & $11.08\substack{+0.06\\-0.07}$ \\
WFI2033 & $21.78\substack{+0.28\\-0.23}$ & $19.07\substack{+0.35\\-0.26}$ & $19.6\pm4.5$ & $0.28\pm0.02$ & $0.52\pm0.01$ & $0.625$ & $10.51\substack{+0.09\\-0.11}$ \\
SDSS1206 & $21.31\substack{+0.23\\-0.19}$ & $18.30\substack{+0.05\\-0.05}$ & $33.3\pm6.4$ & $0.11\pm0.02$ & $4.57\pm0.53$ & $0.625$ & $10.77\substack{+0.08\\-0.09}$ \\
HE1104 & $21.25\substack{+0.16\\-0.14}$ & $19.16\substack{+0.02\\-0.02}$ & $14.0\pm2.0$ & $0.27\pm0.02$ & $1.05\pm0.04$ & $0.625$ & $11.05\substack{+0.06\\-0.07}$ \\
SDSS0246 & $23.44\substack{+0.28\\-0.22}$ & $20.85\substack{+0.08\\-0.07}$ & $4.0\pm0.9$ & $0.44\pm0.08$ & $4.96\pm0.08$ & $0.625$ & $10.75\substack{+0.09\\-0.11}$ \\
HS2209 & $20.72\substack{+0.26\\-0.21}$ & $19.20\substack{+0.04\\-0.04}$ & $12.5\pm2.7$ & $1.96\pm1.28$ & $3.15\pm1.40$ & $1.00$ & $11.04\substack{+0.08\\-0.10}$ \\
HE0047 & $22.92\substack{+0.48\\-0.33}$ & $20.37\substack{+0.20\\-0.17}$ & $2.3\pm0.8$ & $0.32\pm0.15$ & $4.18\pm0.75$ & $0.625$ & $10.91\substack{+0.13\\-0.19}$ \\
\hline
\end{tabular}
%}
\begin{tablenotes}
      \small
      \item Note: $-$ Inference of the host galaxy properties. Column (2)-(6): photometry derived using the imaging data is listed Table~\ref{data_set}. Column (7): \pink{adopted age of stellar population with solar metallicity.} Column (8): inferred stellar mass. 
\end{tablenotes}    
\end{threeparttable}
\end{table*}


\section{Results}\label{sec:result}
In this section, we describe the approaches and assumptions used to estimate the stellar populations in our sample. Then, we adopt stellar population templates to derive the stellar mass. We study the \mbh-\mstar\ relations and compare our measurements with ones taken from the literature.

\subsection{Stellar population and mass}\label{sec:mstar}
%[] Color inference for the first three cases to decide their stellar template. 
Besides the imaging data analyzed in the last section, some of the systems have been also imaged by \hst\ through other bands, providing color information. Given that we have used the highest signal-to-noise ratio data for our primary models,  the analysis of the other bands has lower fidelity and we use it only to infer the colors so as to assist in the estimation of the stellar mass (i.e., Table~\ref{tab:host_measure} column (2)).

{\bf HE0435} ~ In addition to the F160W data, this system has been observed by \hst\ through filters F814W and F555W. The multi-band imaging data provides us with the opportunity to estimate the host color and improve the stellar mass estimate. We adopted the lensing modelling approach described in Section~\ref{sec:photometry} to infer the host photometry through the F814W and F555W bands. Based on the three band host magnitudes, we find that a 1.5~Gyr age and solar metallicity stellar population provides a good match to the color. More detailed information is presented in Appendix~\ref{app:HE0435}.
%Considering that the lensed host light in the image plane can be considered as the subtraction of the AGNs and deflectors light which should be less dependent on the lens model, we expect its magnitude in the image plane could be obtained with higher accuracy. Thus, we infer the color of the lensed arcs in the image plane. Based a their magnitudes in three bands, we find a 1.5 Gyr stellar population with solar metallicity could well match this color. More detailed information is presented in Appendix~\ref{app:HE0435}.}

{\bf RXJ1131} ~ The host galaxy of RXJ1131 is lensed to a very extended arc in the image plane. The spectral energy distribution of the arc can be directly inferred in the image plane, since lensing is achromatic. Based on the \hst\ imaging data through the three filters F814W and F555W and F160W, we adopt the SED estimated by~\citet{Ding2017b} with stellar populations of 3 Gyr and 1.5 Gyr (solar metallicity) for its bulge and disk, respectively. \pink{We refer the interested reader to that paper for more details.}

{\bf WFI2033} ~ In addition to the F160W filter, images through the four filters F125W, F140W, F555W, F814W are also available in the \hst\ archive. As for HE0435, we infer the host color in the image plane. We find a stellar population with 0.625 Gyr is a good match to the colors (more details see Appendix~\ref{app:WFI2033}).

{\bf HE1104 and the other systems} ~ HE1104 has also been observed by \hst\ through the F555W and F814W bands. However, given the limited exposure time in these two bands, the lensed arcs is too faint to be detected; thus, we are not able to infer the color of the host for HE1104. No multi-band information is available for the other systems. Thus, for these cases, we follow D20 and assume a typical stellar population age, i.e. 1 Gyr and 0.625 Gyr for systems as $z<1.44$ and $z>1.44$, respectively. Of course this choice is not unique. However, we stress that as we are mostly interested in the comparison with D20 this strategy is meant to minimize differences between the two approaches.

A summary of the adopted stellar population ages is given in Table~\ref{tab:host_measure}, column~(7). Applying these templates to the filter magnitudes obtained in last section, we derive the stellar mass of our system, Table~\ref{tab:host_measure}, column~(8). Considering the simulations by~\citet{Ding2017a} and the fact that we are able to obtain very consistent host magnitudes for the HE0435, RXJ1131 and WFI2033 using the independent approaches, the fidelity of the inferred magnitude is expected to be well characterized by the quoted uncertainties. We adopt a typical uncertainty of 0.2 dex for the \mstar\ estimates.

\subsection{The \mbh-\mstar\ relation}\label{sec:relation}
In Figure~\ref{fig:scaling_relation}-(left) we plot \mbh\ vs \mstar\ for our sample, together with the comparison sample introduced in Section~\ref{sec:sample_select}.  We expect that the uncertainty of the \mbh\ (i.e., $0.4~$dex) dominates the error budget for the entire sample. Following D20, we adopt, as baseline, a local \mbh-\mstar\ of the form:
\begin{eqnarray}
\label{eq:MMlocal}
\log \big( \frac{\mathcal M_{\rm BH}}{10^{7}M_{\odot}})= \alpha + \beta \log(\frac{M_*}{10^{10}M_{\odot}}),
  \end {eqnarray}
with  $\alpha = 0.27$, $\beta = 0.98$ based on the local sample of 55 objects measured in a consistent manner. We find that our lensed systems are above the local relation, consistent with the inference from the 32  non-lensed AGNs published by D20 in a  similar redshift range. To quantify the evolution as a function of redshift, we parameterize it as:
\begin{eqnarray}
\label{eq:offset}
\Delta \log \mathcal M_{\rm BH}= \gamma \log (1 + z),
\end{eqnarray} 
where $\Delta \log \mathcal M_{\rm BH}$ is the offset of \mbh\ with respect to the local baseline at fixed \mstar. To make a direct comparison, we reproduce the plot shown in Figure~8 of D20. Then, we add our new lensed AGN measurements and show the result in Figure~\ref{fig:scaling_relation}-(right). We find that the offset from the local relationship is similar for the two samples. Based on Equation~\eqref{eq:offset}, we fit the evolution of the eight lensed systems and obtain $\gamma=1.49\pm0.43$, which agrees with the results of D20 ($\gamma=1.03\pm0.25$ using 32 AGN) within $1-\sigma$ level. The consistency between the two measurements provides an important verification of the accuracy of the results and strengthens the conclusions drawn by D20 that the observed value of \mbh\ at a fixed \mstar\ tends to be larger at higher redshift than the ones in the local universe.

The lensed AGN systems analyzed in this work were selected for time-delay cosmography work, based on the availability of a time delay and the known detectability of the host galaxy. This is a different selection function that adopted by D20. It is thus reassuring that the two samples present the same apparent evolution. Naturally, inferring the true underlying evolution requires modelling the selection function (as done by D20), which is not justified given the small sample size of lensed quasars sample. At this stage, the lensed quasars should thus be considered as a check on possible systematic measurements error in the D20 analysis rather than a stand-alone measurement.


\begin{figure*}
\centering
\begin{tabular}{c c}
{\includegraphics[height=0.45\textwidth]{fig/MBH-Mstar.pdf}}&
{\includegraphics[height=0.45\textwidth]{fig/MBH-Mstar_vz.pdf}}\\
\end{tabular}
\caption{\label{fig:scaling_relation} 
(left): BH masses vs. stellar mass correlation (\mbh-\mstar). The black line and the gray shaded region indicate the best-fit and 1$\sigma$ confidence interval of the form given by Equation~\ref{eq:MMlocal}.
(right): Offset of $\log(\mathcal M_{\rm BH})$ at fixed \mstar as a function of redshift. Our new measurements are overlaid on samples of non-lensed AGNs taken from the literature and measured in a self consistent way to facilitate a direct comparison. The red line and shaded region are the best-fit and $1-\sigma$ results of fitting  Equation~\eqref{eq:offset} to the 32 high-z AGNs measured by D20. The blue line and region are the results based on the eight lensed systems presented in this work. The blue and red areas are in agreement within the errors.}
\end{figure*} 

\section{Systematic Errors}\label{sec:diss}

We considered a set of modelling choices to perform the fitting and the final inference is based on a weighting of top ranked choices. In particular, we treat the two different modelling approaches, i.e., {\it noise boost} and {\it PSF iteration} equally, to derive the averaged results. Of course, the results are somewhat dependent on the weighting scheme, for example, the dispersion of the results by the top ranked choices. However, the dependency is smaller than other sources of uncertainty. Thus, using different weighting schemes would only change the results marginally ($<0.1$~dex).

\pink{In this work, we used a range of mass slope values (i.e., 1.9, 2.0, 2.1) to perform the lens modelling. Then, we used our weighting algorithm introduced in Section~\ref{sec:Modelling} to estimate the systematic uncertainty of our inference and assumed it covers the truth. We apply this method to the entire systems to allow the self-consistency within our sample, despite that four systems (HE0435, RXJ1131, WFI2033, and SDSS1206) have been analyzed by H0LiCOW collaboration with slope value provided. For a sanity check, we calculate the weighted slope value and make a direct comparison to the H0LiCOW's inference. The results are the following (here {\it V.S.} H0LiCOW): HE0435~($2.032\pm 0.07$ {\it V.S.} $1.93\pm0.02$), RXJ1131~($2.02\pm 0.06$ {\it V.S.}$1.95\pm0.045$), WFI2033~($1.94\pm 0.07$ {\it V.S.} $1.95\pm0.02$), SDSS1206~($1.98\pm 0.06$ {\it V.S.} $\sim1.95$). These consistent results demonstrate the efficiency of the systematic uncertainty estimates in this work.}

\pink{To estimate the host stellar mass, we use \sersic\ model to measure the total brightness of the host galaxy. The \sersic\ profile is relatively simple and smooth, which could not capture the clumps in the host galaxy, which lead to the high residuals as appeared in the fittings in Figure~\ref{fig:image_inference}. However, as mention in Section~\ref{subsec:HE0435}, our goal to derive a self-consistent one-step inference of the host properties to make comparison with the measurements of the non-lensed AGN samples, which are measured using the same manner. Note that the total flux of a galaxy would be relatively settled, once a surface brightness profile is chosen. Thus, the inferred host light is considered to be reliable.
}

In addition, we adopt simple stellar populations to derive the stellar mass. For 5/8 systems in our sample we did not have color information, and we used instead a fixed age depending on redshift. The lack of color information by necessity increases the error budget in the inferred \mstar. \pink{Once again, the same approach is adopted in D20, and thus our stellar masses are consistent with the D20's 32 high redshift AGN regarding this point.}

\pink{We do not expect foreground extinction to be significant, because the deflectors are all massive elliptical galaxies. As a sanity check, \citet{1999ApJ...523..617F} and \citet{2008A&A...485..403O} report estimates for HE0435, WFI2033 and HE1104 through filter F160W. For HE1104 the estimated extinction is negative, while for HE0435 and WFI2033 the authors do not report an extinction value because standard extinction laws did not fit the data.
If we focus on the 13 ellipticals from~\citet{1999ApJ...523..617F}, the total median extinction is -0.03, which justifies our point.
We interpret this as due to the small effect by dust being overshadowed by chromatic microlensing or variability.}

\pink{Most importantly, the uncertainty of the \mstar\ is smaller than the estimated uncertainty in the black hole mass (0.4~dex), which dominates the overall error budget.}

\pink{In this work, some options and assumptions have been made to measure the evolution of \mstar-\mbh. For example, a Chabrier IMF was assumed to measure the stellar mass, rather than the Salpeter one. To compare our high redshift measurements with the local ones, we adopt the local sample from~\citet{Bennert++2011, H+R04}, rather than the other local ones in the literature. We adopt our own recipes to calibrate the \mbh. Of course, different options would shift the absolute value of the inferred \mstar\ and \mbh. However, since the entire sample is self-consistent, a different assumption would only shifts the global \mstar-\mbh\ together, leaving the offset value and the evolution conclusion the same. More details can be found in D20, Section 6.}

\section{Conclusions}\label{sec:con}
Using eight strongly lensed AGN systems, we presented a new measurements of the correlations of the mass of supermassive black hole with the stellar masses of their host galaxy. We adopt \pink{up-to-date} lens modelling techniques to estimate the magnitude host galaxy, in terms of a standard \sersic\ profile. We estimated the \mbh\ of our sample using a set of calibrated single-epoch estimators to assure self-consistency and consistency with the comparison samples taken from the literature.

We directly compare our sample to the recent measurements by~\citet[][D20]{Ding2020a}, who used the same approach to derive the \mstar\ and calibrated the \mbh\ using consistent recipes. The \mbh-\mstar\ correlation and its evolution with cosmic time is in excellent agreement with the results obtained by D20, as shown in Figure~\ref{fig:scaling_relation}. Thus, the present analysis further strengthens the conclusion of D20, which can be summarized as follows. First, the growth of the supermassive black hole predates that of its host galaxy during their co-evolution, even when considering total stellar mass, as it is often done at high redshift. However, the actual morphologies of these hosts are likely to be more complex, including a disk and bulge component, as in the case of  RXJ1131. In contrast, the galaxies in the local sample are typically bulge-dominated and the bulge mass is adopted as their  \mstar. Thus, the reported evolution is weaker than what would be inferred by comparing \mbh\ to bulge \mstar\ at all redshifts \citep{Bennert++2011}. Taken together, these results are consistent with a scenario in which the stellar mass is transferred from disk to the bulge at a faster rate than the growth of \mbh\ since $z\sim2$.

Our work based on highly magnified AGN showcases the power of strong lensing to effectively increase the resolution of a telescope and shows that uncertainties related to lens modelling are subdominant with respect to other sources of uncertainty like black hole mass. Furthermore, our work provides a powerful verification of the fidelity of the host galaxy reconstruction in non-lensed AGN. 

\pink{The strong lensing can even help to extend the measurements of the AGN host to higher redshift. It is worth noting that the lensed AGN adopted by this study are adopted from the H0LiCOW project which only focuses on the moment of the peak of the AGN luminosity function. In fact, the observation of \hst\ could even unveil the lensed AGN host of  BRI0952$-$0115 at $z=4.5$.}

In conclusion, lensed AGNs have great potential to extend to study of the \mbh-\mstar\ correlation at even higher redshifts than those considered here.  So far, this kind of work has been limited by sample size. However, given the pace of discovery of lensed quasars in imaging and spectroscopic surveys ~\citep[e.g.,][]{Oguri2010, Agn++15,Mor++16,Sch++16,Ost++17,Tre++18, Ang++18, Lem++20}, the samples of lensed AGNs with hosts that can be recovered with high fidelity is likely to continue to grow in wide field imaging and spectroscopic surveys. The forthcoming launch of the {\it James Webb Space Telescope} and the first light of adaptive optics-assisted extremely large telescopes may provide high-quality imaging data of AGNs at higher redshift (up to $z\sim7$).

\section*{Acknowledgements}
\pink{We are grateful to Frederic Courbin, Leon Koopmans for useful comments
and suggestions that improved this manuscript.}

This work is based in part on observations made with the NASA/ESA Hubble Space Telescope, obtained at the Space Telescope Science Institute, which is operated by the Association of Universities for Research in Astronomy, Inc., under NASA contract NAS 5-26555. X.D., S.B., and T.T. acknowledge support by the Packard Foundation through a Packard Research fellowship to T.T. \pink{This work is supported by JSPS KAKENHI Grant Number JP18H01251 and the World Premier International Research Center Initiative (WPI), MEXT, Japan. SHS thanks the Max Planck Society for support through the Max Planck Research Group.} 

\pink{This work has made use of \lenstronomy~\citep{lenstronomy}, {\sc Astropy}~\citep{Astropy}, {\sc photutils}~\citep{photutils}, {\sc Matplotlib}~\citep{Matplotlib} and standard Python libraries.}

%%%%%%%%%%%%%%%%%%%%%%%%%%%%%%%%%%%%%%%%%%%%%%%%%%

%%%%%%%%%%%%%%%%%%%% REFERENCES %%%%%%%%%%%%%%%%%%

% The best way to enter references is to use BibTeX:

\bibliographystyle{mnras}
\bibliography{reference} % if your bibtex file is called example.bib


% Alternatively you could enter them by hand, like this:
% This method is tedious and prone to error if you have lots of references
%\begin{thebibliography}{99}
%\bibitem[\protect\citeauthoryear{Author}{2012}]{Author2012}
%Author A.~N., 2013, Journal of Improbable Astronomy, 1, 1
%\bibitem[\protect\citeauthoryear{Others}{2013}]{Others2013}
%Others S., 2012, Journal of Interesting Stuff, 17, 198
%\end{thebibliography}
%%%%%%%%%%%%%%%%%%%%%%%%%%%%%%%%%%%%%%%%%%%%%%%%%%

%%%%%%%%%%%%%%%%% APPENDICES %%%%%%%%%%%%%%%%%%%%%

\appendix

\section{Estimate of the color of the host}
In this section, we describe the details of how we select the stellar population of the host used in the computation of the stellar mass estimate.

\subsection{HE0435}\label{app:HE0435}
In addition to WFC3/F160W, HE0435 has been observed through bands ACS/F814W and ACS/F555W (GO-9744; PI: C. S. Kochanek). Our aim is to derive the brightness of the host galaxy through all the three band to investigate the color and stellar population in the image plane. We use the modelling approach introduced in Section~\ref{sec:photometry} to perform this inference for the F814W and F555W bands. To save computer time, we only use the {\it PSF iteration} approach. We also fix the lens mass slope value to 1.9 since it is closer the inference by~\citet[][i.e., $\gamma\sim1.93$]{Wong2017}. The inference of the fittings are shown in Figure~\ref{fig:app_HE0435}-(a), (b). Having obtained the magnitude of the lensed host in the three bands, we use the {\sc Gsf} package\footnote{\url{https://github.com/mtakahiro/gsf}}~\citep{Morishita2019} to perform the SED fitting. \pink{A set of age grids (up to 3.0 Gyr) are used in this fitting with a constant SFR and a flexible form for star formation histories. Note that there is a degeneracy in the age and metallicity; however, this degeneracy has little affect on the \mstar\ inference. In this work, we fix the metallicity to infer the age, and the uncertainty of the age is not considered. Finally, a} stellar population with 1.5~Gyr of age and solar metallicity provides the best fit to the colors, as shown in Figure~\ref{fig:app_HE0435}-(c). This stellar population is used to estimate the host stellar mass.


\begin{figure*}
\centering
\subfloat[HE0435 F555W]{\includegraphics[ width=0.8\textwidth]{fig/HE0435_f555w_inference.pdf}}\\
\subfloat[HE0435 F814W]{\includegraphics[ width=0.8\textwidth]{fig/HE0435_f814w_inference.pdf}}\\
\subfloat[Host galaxy stellar template inference]{\includegraphics[ width=0.5\textwidth]{fig/HE0435_host_color.pdf}}\\
\caption{\label{fig:app_HE0435} 
Illustrations of the inference of HE0435 using the multi-band data. Top two panels: best-fit model of the lensed arcs shown as Figure~\ref{fig:image_inference}. Bottom panel: SED inference. The red points with error bars represent the inferred image plane host flux in the three bands.}
\end{figure*} 

%\subsection{RXJ1131}\label{app:RXJ1131}
%Using the three band data including  ACS/F814W and ACS/F555W and NIC2/F160W (GO-9744; PI: C. S. Kochanek), we preform the SED fitting in the imaging plane directly pixel by pixel in Figure~\ref{fig:app_RXJ1131}. \ding{This part is actually done by Takahiro in~\citep{Ding2017b}. The image of the fitted age looks noisy, but it is in the range of the [9.25 - 9.50] (logAge), i.e. [1.5, 3.0] (Gyr). We can remove this A2 but only mention the fitting is done 2017 in that paper.}
%\begin{figure}
%\centering
%{\includegraphics[width=0.25\textwidth]{fig/map_age_zRXJ_1131.pdf}}\\
%\caption{\label{fig:app_RXJ1131} 
%Illustrations of the SED fitting of the galaxy age pixel by pixel.  }
%\end{figure} 

\subsection{WFI2033}\label{app:WFI2033}
In addition to WFC3/F160W, WFI2033 has been observed by \hst\ in  four bands, i.e. WFC3/F125W (GO-12874; PI: D. Floyd), WFC3/F140W (GO-13732; PI: A. Nierenberg), ACS/F555W+F814W  (GO-9744; PI: C. S. Kochanek). Similar to Section~\ref{app:HE0435}, the lens modelling and the host photometry of WFI2033 have been inferred through these four bands. The results are presented in the Figure~\ref{fig:app_WFI2033}-(a)$-$(d). Unfortunately, the  F125W data are too shallow to robustly detect the host.  Thus, we does not use the F125W band for the SED fitting. Using four-band photometry, we find a stellar population of 0.625 Gyr age provides a good match to the colors, as shown in Figure~\ref{fig:app_WFI2033}-(e).
\begin{figure*}
\centering
\subfloat[WFI2033 F125W]{\includegraphics[ width=0.8\textwidth]{fig/WFI2033_f125w_inference.pdf}}\\
\subfloat[WFI2033 F140W]{\includegraphics[ width=0.8\textwidth]{fig/WFI2033_f140w_inference.pdf}}\\
\subfloat[WFI2033 F555W]{\includegraphics[ width=0.8\textwidth]{fig/WFI2033_f555w_inference.pdf}}\\
\subfloat[WFI2033 F814W]{\includegraphics[ width=0.8\textwidth]{fig/WFI2033_f814w_inference.pdf}}\\
\subfloat[Host galaxy stellar template inference]{\includegraphics[ width=0.5\textwidth]{fig/WFI2033_host_color.pdf}}
\caption{\label{fig:app_WFI2033} 
Same as Figure~\ref{fig:app_WFI2033} but for WFI2033. The F125W dare are too shallow to detect the host and are thus omitted in the fit.}
\end{figure*} 
%%%%%%%%%%%%%%%%%%%%%%%%%%%%%%%%%%%%%%%%%%%%%%%%%%


% Don't change these lines
\bsp	% typesetting comment
\label{lastpage}
\end{document}

% End of mnras_template.tex